\documentclass[11pt]{scrartcl}
\title{An Infrastructure to Store and Analyse Seismic Data as Suffix Trees}
\subtitle{MSc Data Analytics - Project Proposal}
\date{April 2016}
\author{Tom Taylor}

\usepackage{graphicx}
\graphicspath{ {img/} }
\setlength{\parindent}{4em}
\setlength{\parskip}{1em}

\begin{document}
\maketitle
\begin{itshape}
	\noindent This report is substantially the result of my own work except where explicitly
	indicated in the text. I give my permission for it to be submitted to the JISC
	Plagiarism Detection Service. I have read and understood the sections on plagiarism
	in the Programme Handbook and the College website.
	
	\noindent The report may be freely copied and distributed provided the source is explicitly
	acknowledged.
\end{itshape}

\tableofcontents

\newpage

\section{Abstract}
	The purpose of this project is to develop an infrastructure and tool set for converting raw seismic time series data in to a searchable string and then to store this data as a suffix tree for fast searching and analysis.  An interface would then be developed to enable the searching of these suffix trees and provide visualisation of the data.
	
\section{Background}
	Seismic waves are recorded as movement over three axis: vertical (hereafter referred to as \textbf{z}) alongside horizontal in terms of north-south(\textbf{n}) and east-west(\textbf{e}).
	
	Some words about SAX.
	
	Some words about Suffix Trees.

\section{Objectives}
\begin{enumerate}
	\item Interpretation of raw data from seismic stations and storage as time-series data in a time-series database such as OpenTSDB for easy access during conversion to SAX and for later rendering during interactions with the data.
	\item Conversion of the data to SAX and storage in a suffix tree.
	\item An interface for searching and viewing the raw data.
\end{enumerate}

\section{Approach}

\section{Plan}


\end{document}